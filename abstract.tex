%% The following is a directive for TeXShop to indicate the main file
%%!TEX root = diss.tex

\chapter{Abstract}

The duality approach refers to a class of optimization techniques tackling the "dual" problem that arises from the original problem. Numerous notable improvements in strengthening the duality approach have been accomplished in the last two decades due to its superior performance in tackling many large-scale optimization problems. In this thesis, we investigate and extend the duality approach to two classes of optimization problems: structured optimization, whose solution exhibit a specific structure, and federated learning, which aims to learn a model from decentralized data sources collaboratively. Specifically, in the first part of this thesis, we characterize the dual correspondence in structured optimization. We further show that this dual correspondence allows us to develop efficient algorithms and design new models for structured optimization. In the second part of this thesis, we tackle the federated optimization problem from a dual perspective and propose a dual approach. We demonstrate theoretically and empirically that our approach enjoys better convergence performance than those primal-based approaches under specific scenarios. Besides, we also explore some application scenarios for structured optimization in federated learning. In the third part of this thesis, we study the problem of evaluating clients' contributions in federated learning. We propose fair and efficient contribution valuation metrics for both horizontal and vertical federated learning, where structured optimization plays a crucial role in our design. 








